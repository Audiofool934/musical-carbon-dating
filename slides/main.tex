\documentclass[aspectratio=169, 10pt]{beamer}

% --- Theme & Colors ---
\usetheme{Madrid}
\usecolortheme{dolphin}
\usefonttheme{professionalfonts}

% --- Packages ---
\usepackage{bookmark}
\usepackage{booktabs}
\usepackage{graphicx}
\usepackage{amsmath, amssymb}
\usepackage{hyperref}
\usepackage{ragged2e}
\usepackage{xcolor}

% --- Metadata ---
\title[Musical Carbon Dating]{Musical Carbon Dating}
\subtitle{A Regression Analysis of Music Evolution (1960--2020)}
\author[Audiofool]{Audiofool}
\institute[Data Science]{University Statistical Analysis Project}
\date{\today}

% --- Configuration ---
\setbeamertemplate{navigation symbols}{}
\setbeamerfont{caption}{size=\footnotesize}
\graphicspath{{../output/figures/}}

% --- Custom Commands ---
\newcommand{\hl}[1]{\textbf{\textcolor{blue}{#1}}}

\begin{document}

% -----------------------------------------------------------------------------
% Title Slide
% -----------------------------------------------------------------------------
\begin{frame}
    \titlepage
\end{frame}

\begin{frame}{Table of Contents}
    \tableofcontents
\end{frame}

% -----------------------------------------------------------------------------
% Section 1: Introduction
% -----------------------------------------------------------------------------
\section{Introduction}

\begin{frame}{The Research Question}
    \begin{block}{The Arrow of Time}
        \centering
        \textit{"Can we define the 'Arrow of Time' for music using only audio signal properties?"}
    \end{block}
    
    \vspace{0.5cm}
    \textbf{Objective}: 
    To quantify the evolution of musical production styles from 1960 to 2020 using Regression Analysis.
    
    \vspace{0.3cm}
    \textbf{Why it matters}:
    \begin{itemize}
        \item \textbf{Commercial}: Recommendation systems (Spotify) need to understand "vintage" vs "modern" aesthetics beyond just metadata.
        \item \textbf{Cultural}: Quantifying the impact of the Digital Revolution (1999).
    \end{itemize}
\end{frame}

\begin{frame}{The Dataset}
    \begin{columns}
        \column{0.5\textwidth}
        \textbf{Source}: Spotify 600k Tracks Dataset.
        
        \vspace{0.3cm}
        \textbf{Filtering Criteria}:
        \begin{itemize}
            \item Year: $1960 \le T \le 2020$.
            \item Popularity $> 30$ (Focus on culturally relevant tracks).
            \item Cleaned $N = \mathbf{250,971}$.
        \end{itemize}
        
        \vspace{0.2cm}
        \tiny \textbf{Source}: \href{https://www.kaggle.com/datasets/yamaerenay/spotify-dataset-19212020-600k-tracks}{Yamac Eren Ay (Kaggle, 2021)}
        
        \vspace{0.3cm}
        \textbf{Features ($p=13$)}:
        \begin{itemize}
            \item \textbf{Physical}: \texttt{loudness}, \texttt{tempo}, \texttt{duration}.
            \item \textbf{Perceptual}: \texttt{acousticness}, \texttt{energy}, \texttt{valence}.
            \item \textbf{Musical}: \texttt{key}, \texttt{mode}.
        \end{itemize}

        \column{0.5\textwidth}
        \begin{figure}
            \includegraphics[width=0.9\linewidth]{figures/eda_correlation_heatmap.png}
            \caption{Feature Correlation Matrix (Note $r_{loud, energy} \approx 0.7$)}
        \end{figure}
    \end{columns}
\end{frame}

% -----------------------------------------------------------------------------
% Section 2: Methodology (Foundations)
% -----------------------------------------------------------------------------
\section{Methodology: Linear Foundations}

\begin{frame}{Phase II: Simple Linear Regression (The "Loudness War")}
    We started with a single predictor: \textbf{Loudness}.
    $$ y_i = \beta_0 + \beta_{loud} x_{i,loud} + \varepsilon_i $$
    
    \begin{alertblock}{Result}
        \begin{itemize}
            \item $R^2 = 0.144$.
            \item $t$-statistic $= 183.7$ ($p < 0.001$).
            \item \textbf{Interpretation}: Music has gotten significantly louder over time (+1.2 years per dB).
        \end{itemize}
    \end{alertblock}
\end{frame}

\begin{frame}{Phase III: Multiple Linear Regression (MLR)}
    We expanded to the Full Model ($p=13$):
    $$ \mathbf{y} = \mathbf{X}\boldsymbol{\beta} + \boldsymbol{\varepsilon} $$
    
    \begin{columns}
        \column{0.6\textwidth}
        \textbf{Model Performance}:
        \begin{itemize}
            \item $R^2 = 0.296$.
            \item Test RMSE $\approx 12.06$ years.
        \end{itemize}
        
        \vspace{0.2cm}
        \textbf{Key Insights}:
        \begin{itemize}
            \item \texttt{Acousticness}: Strong negative trend ($\beta \approx -2.8$).
            \item \texttt{Danceability}: Positive trend ($\beta \approx +22$).
        \end{itemize}

        \column{0.4\textwidth}
        \begin{figure}
            \includegraphics[width=\linewidth]{figures/partial_regression_partial_regression_-_acousticness.png}
            \caption{Partial Regression: Acousticness}
        \end{figure}
    \end{columns}
\end{frame}

% -----------------------------------------------------------------------------
% Section 3: Diagnostics
% -----------------------------------------------------------------------------
\section{Diagnostics: The Audit}

\begin{frame}{Phase IV: Diagnostics Overview}
    We rigorously tested the Gauss-Markov assumptions.
    
    \begin{table}[]
        \centering
        \begin{tabular}{llc}
            \toprule
            \textbf{Assumption} & \textbf{Test Used} & \textbf{Outcome} \\
            \midrule
            \textbf{Linearity} & Partial F-Test ($x_{dur}^2$) & \textcolor{red}{Reject ($F=304.7$)} \\
            \textbf{Homoscedasticity} & Breusch-Pagan & \textcolor{red}{Reject ($LM=19391$)} \\
            \textbf{Multicollinearity} & VIF Score & \textcolor{green}{Pass (Max $< 4.0$)} \\
            \bottomrule
        \end{tabular}
        \caption{Diagnostic Summary}
    \end{table}
\end{frame}

\begin{frame}{Residual Analysis: Boundedness \& Heterogeneity}
    \begin{columns}
        \column{0.6\textwidth}
        \begin{figure}
            \includegraphics[width=\linewidth]{figures/residuals_mlr_residuals.png}
            \caption{Residuals vs Fitted (Note the diagonal boundaries)}
        \end{figure}
        
        \column{0.4\textwidth}
        \textbf{Boundedness Artifacts}:
        \begin{itemize}
            \item $Y \in [1960, 2020]$ creates diagonal ceilings/floors ($e = y - \hat{y}$).
            \item Prediction space compressed at boundaries.
        \end{itemize}
        
        \vspace{0.2cm}
        \textbf{Heavy Tails (Q-Q Plot)}:
        \begin{itemize}
            \item Non-normal errors $\neq$ Model failure.
            \item Indicates \textbf{Stylistic Heterogeneity} (Retro/Futuristic outliers).
        \end{itemize}
        
        \vspace{0.2cm}
        \textit{Remedy}: Robustness via CLT ($N=250k$). Non-normality justifies the \textbf{Nostalgia Index}.
    \end{columns}
\end{frame}

% -----------------------------------------------------------------------------
% Section 4: Structural Break
% -----------------------------------------------------------------------------
\section{The Structural Break (1999)}

\begin{frame}{Phase VI: The Digital Revolution}
    We hypothesized a structural break at $T=1999$ (Napster \& ProTools Era).
    $$ y_i = \mathbf{x}_i^\top \boldsymbol{\beta} + \delta D_i + D_i (\mathbf{x}_i^\top \boldsymbol{\gamma}) + \varepsilon_i $$
    where $D_i = \mathbb{I}(Year \ge 1999)$.
    
    \begin{alertblock}{Explanatory Result}
        \begin{itemize}
            \item \textbf{$R^2$ Variance Explained}: $0.296 \to \mathbf{0.734}$ (Explanatory).
            \item \textbf{Interpretation}: This reflects the model's ability to fit the data once the "Digital Era" is accounted for.
            \item \textbf{Conclusion}: The mechanism of music creation fundamentally changed in 1999.
        \end{itemize}
    \end{alertblock}
\end{frame}

\begin{frame}{The "Scissor Effect" (Interaction)}
    \begin{columns}
        \column{0.5\textwidth}
        \textbf{Acousticness Coefficient Flip}:
        \begin{itemize}
            \item \textbf{Pre-1999}: $\beta \approx -2.8$ (Folk/Rock era).
            \item \textbf{Post-1999}: $\beta + \gamma \approx +0.9$.
            \item \textbf{Meaning}: In the digital era, acoustic elements became a stylistic choice (e.g., "Unplugged") rather than a technological limitation.
        \end{itemize}
        
        \column{0.5\textwidth}
        \begin{figure}
            \includegraphics[width=\linewidth]{figures/scissor_plot.png}
            \caption{Unified Scissor Plot: LOWESS Trend}
        \end{figure}
    \end{columns}
\end{frame}

% -----------------------------------------------------------------------------
% Section 5: Model Selection
% -----------------------------------------------------------------------------
\section{Model Selection}

\begin{frame}{Phase V: Seeking Parsimony}
    We compared algorithms to select the optimal feature set.
    
    \begin{table}[]
        \centering
        \begin{tabular}{llcl}
            \toprule
            \textbf{Method} & \textbf{Criterion} & \textbf{Features} & \textbf{Verdict} \\
            \midrule
            Full Model & None & 13 & Baseline \\
            \textbf{LASSO} & $L_1$ Penalty & 12 & \textbf{Preferred} \\
            Stepwise & AIC & 11 & Too aggressive \\
            \bottomrule
        \end{tabular}
    \end{table}
    
    \textbf{Outcome}:
    We retained complex features like \texttt{Instrumentalness} and \texttt{Speechiness} as they carry era-specific signal (e.g., Solos vs Rap).
\end{frame}

% -----------------------------------------------------------------------------
% Section 6: Applications
% -----------------------------------------------------------------------------
\section{Commercial Applications}

\begin{frame}{The "Nostalgia Index"}
    We propose a \textbf{Retro Detector} metric:
    $$ \mathcal{N}_i = \hat{y}_{blind} - y_{actual} $$
    
    \begin{itemize}
        \item $\mathcal{N} \ll 0$: Song sounds older than it is ("Retro").
        \item $\mathcal{N} \gg 0$: Song sounds futuristic ("Avant-garde").
    \end{itemize}
    
    \vspace{0.5cm}
    \begin{table}[]
        \centering
        \small
        \begin{tabular}{lcccc}
            \toprule
            \textbf{Song} & \textbf{Actual} & \textbf{Pred} & \textbf{$\Delta$ (Yrs)} & \textbf{Vibe} \\
            \midrule
            The Weeknd - \textit{Echoes} & 2012 & 1993.1 & \textbf{-18.9} & 90s R\&B \\
            Mark Ronson - \textit{Uptown} & 2015 & 2013.1 & \textbf{-1.9} & Retro-ish \\
            Dua Lipa - \textit{Physical} & 2020 & 2009.0 & \textbf{-11.0} & 80s Synth \\
            \bottomrule
        \end{tabular}
        \caption{Verified Predictions on Test Set}
    \end{table}
\end{frame}

\begin{frame}{Visualizing Prediction Accuracy}
    \centering
    \begin{figure}
        \includegraphics[height=0.6\textheight]{figures/pred_vs_act_final_break_model_predictions.png}
        \caption{Actual vs Predicted (Test Set). Note the tighter fit post-1999.}
    \end{figure}
\end{frame}

% -----------------------------------------------------------------------------
% Conclusion
% -----------------------------------------------------------------------------
\section{Conclusion}

\begin{frame}{Conclusion}
    \begin{enumerate}
        \item \textbf{Success}: We can date music/audio with $RMSE \approx 7$ years (Era-Informed).
        \item \textbf{Discovery}: 1999 was a structural singularity in music history.
        \item \textbf{Utility}: The Nostalgia Index successfully identifies "Retro" hits.
        \item \textbf{Curriculum Alignment}: Fully utilized SLR, MLR, Diagnostics, Ridge/Lasso, and Interaction effects.
    \end{enumerate}
    
    \vspace{1cm}
    \centering
    \Large \textbf{Thank You} \\
    \small \textit{Questions?}
\end{frame}

\end{document}
